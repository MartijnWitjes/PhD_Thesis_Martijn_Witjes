\chapter{Summary}
\label{cha:Summary}

This thesis aims to contribute towards efforts made in large-scale land cover mapping, with an emphasis on the benefits of combining several datasets from different sources and of different types. It presents different steps of a methodology to extract training data from multiple rich human-annotated datasets and overlay them on Earth observation data from diverse sources. It furthermore details the challenges and benefits of creating land cover maps that navigate the trade-off between spatial, temporal, and thematic resolution, as well as quantity and allocation accuracy.

\textbf{Chapter 2} focuses on the benefits and challenges of harmonizing and combining large-scale spatiotemporal datasets for land cover mapping, most of which were used in the following chapters. The chapter details the work that went into creating, harmonizing, and imputing multiple Earth observation datasets (Landsat, Sentinel-2, and a new 30m resolution DTM) covering Europe. It introduces and describes the imputation algorithm TMWM that was used to impute the Landsat data, and validates its accuracy in a spatiotemporally explicit way. It then explores how combining the different datasets improves the accuracy of land cover classification models. Lastly, it shows that models trained on samples from a longer time range can generalize better to years that they have not been trained on.

\textbf{Chapter 3} focuses on the production of annual land use / land cover maps of Europe for 2000-2020. It details the steps taken to harmonize and clean the training data from multiple openly available sources (CORINE, LUCAS) into a legend with 43 classes. A thorough accuracy assessment using cross-validation and an independent set of S2GLC validation points describes how well the model generalizes across space and time, and quantifies the trade-off imposed by having a legend with high thematic resolution. Results show that the maps have similar accuracy as other current continental-scale maps at low thematic resolution, and that a more detailed legend introduces more errors.

\textbf{Chapter 4} introduces IMP, an algorithm that uses land cover area estimates to iteratively classify land cover from existing probabilities, producing maps whose class proportions match the input estimates. It details the algorithm and showcases its use by mapping five European countries in five years. The accuracy of the maps is compared with maps created using highest likelihood classification. Results show that the proportional maps do not only have more accurate class proportions, but equal or better accuracy than highest likelihood maps. We also compare the accuracy and proportions of maps based on probabilities predicted by models trained on data representative of the area of interest, and probabilities predicted by a general model trained on large parts of Europe. Results show that maps based on general model predictions reach more accurate class proportions, while maps based on local model predictions are slightly more accurate. Finally, it presents an unintentional finding that the iterations at which the algorithm classifies certain pixels is related to the accuracy of those pixels, suggesting that it can be used to generate pixel-level accuracy estimates.