%\thispagestyle{empty}
%\begin{flushleft}
%\Large{\textbf{PE$\&$RC PhD Training Certificate}
%\end{flushleft}
\chapter{PE$\&$RC Training and Education Statement }

\begin{minipage}[c]{.55\textwidth}

With the training and education activities listed below the PhD candidate has complied with the requirements set by the C.T. de Wit Graduate School for Production Ecology and Resource Conservation (PE$\&$RC) which comprises of a minimum total of 32 ECTS (= 22 weeks of activities) 
\end{minipage}
\begin{minipage}[c]{.35\textwidth}
\begin{flushright}
\vspace{0pt}\includegraphics[width=4.6cm,height=4.6cm]{PERC_logo.pdf}
\end{flushright}
\end{minipage}%

\bigskip

\textbf{Review / project proposal (4.5 ECTS)}
\begin{itemize}[nolistsep]
    \item Automated machine learning for spatial and spatio-temporal data cubes: predicting land use and land cover and landscape dynamics at regional and global scales.
\end{itemize}

\textbf{Post-graduate courses (7.5 ECTS)}
\begin{itemize}[nolistsep]
    \item Summer school; OpenGeoHub (2020, 2021, 2022)
    \item Uncertainty propagation in spatial environmental modelling; PE\&RC (2022)
    \item Quarto; VLAG (2023)
\end{itemize}

\textbf{Deficiency, refresh, brush-up courses (6.6 ECTS)}
\begin{itemize}[nolistsep]
    \item Geo for Good; Google (2020)
    \item Workshop on big data and artificial intelligence in earth observation; European Commission (2020)
    \item Introduction to computer science; Harvard University (2021)
\end{itemize}

\textbf{Laboratory training and working visits (2.1 ECTS)}
\begin{itemize}[nolistsep]
    \item Species distribution modelling with deep learning; BIOMAC, Amsterdam, the Netherlands (2021)
    \item Combining land cover mapping with official statistics; Eurostat, Brussels, Belgium (2023)
    \item OEMC Co-development sessions; WUR \& OpenGeoHub (2023)
\end{itemize}

\textbf{Competence, skills and career-oriented activities (2.6 ECTS)}
\begin{itemize}[nolistsep]
    \item Project and time management; WGS (2021)
    \item PhD Workshop carousel; PE\&RC (2023)
    \item Mindful productivity; WGS (2023)
    \item Reviewing a scientific manuscript; WGS (2023)
\end{itemize}

\textbf{Scientific integrity/ethics in science activities (0.3 ECTS)}
\begin{itemize}[nolistsep]
    \item Scientific integrity; WGS (2021)
\end{itemize}

\textbf{PE\&RC Annual meetings, seminars and the PE\&RC weekend \\ (2.1 ECTS)}
\begin{itemize}[nolistsep]
    \item PE\&RC Weekend for first years (2021)
    \item Midterm retreat (2022)
    \item PE\&RC day (2022)
\end{itemize}

\textbf{Discussion groups / local seminars / other scientific meetings (4.5 ECTS)}
\begin{itemize}[nolistsep]
    \item GeoHarmonizer scientific meetings (2020--2021)
    \item SoilMacroFauna discussion groups \& meetings (2021--2024)
    \item OpenEarthMonitor seminars (2022--2024)
\end{itemize}

\textbf{International symposia, workshops and conferences (3.8 ECTS)}
\begin{itemize}[nolistsep]
    \item Open Data Science Europe workshop; oral presentation; Wageningen, the Netherlands (2021)
    \item SoilMacroFauna Workshop; oral presentation; Leipzig, Germany (2022)
    \item Living Planet Symposium; oral and poster presentation; Bonn, Germany (2022)
\end{itemize}

\textbf{Societally relevant exposure (0.6 ECTS)}
\begin{enumerate}[nolistsep]
    \item Guest lecture at primary school: satellites, land cover, climate change (2023)
    \item Guest lecture at CineScience, Heerenstraat Theater Wageningen: Land cover \& climate change (2024)
\end{enumerate}

\textbf{Lecturing / supervision of practicals / tutorials (4.8 ECTS)}
\begin{itemize}[nolistsep]
    \item Geoscripting: Using Python for Geospatial data (2021)
    \item Supervision of Msc student project on gap filling methods; Statistical University of Muenchen, Germany (2021)
    \item Land cover classification with deep learning (2022)
    \item Using STAC and scikit-map for land cover classification (2023)
\end{itemize}
